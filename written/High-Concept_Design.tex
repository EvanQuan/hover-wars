% \documentclass[12p]{article}
\documentclass{article}
% bibliographystyle{seg asdf}
% \tiny\bibligoraphy{seq_eg} .bib file
% \usepackage[margin=1in, headheight=110pt]{geometry}
\usepackage[letterpaper, margin=1in]{geometry}
\usepackage{amssymb, amsmath, amsfonts, amsthm}
\usepackage{mathpazo}
\usepackage{setspace}
% \usepackage{probsoln}
\usepackage{fancyhdr}
\usepackage{hyperref}
\usepackage{float}
\usepackage{tikz}
\usepackage{enumitem}
\usepackage{listings}
% \usepackage{lipsum}
% \usepackage{parskip} % Use for extra line spacing

\setlength{\parindent}{0 in}

% TODO change these when we figure them out
\newcommand{\name}{Project A}
\newcommand{\team}{Team A}
\newcommand{\botcount}{4}

% TODO change the placement of these?
\pagestyle{fancy}
\lhead{\name{}}
\rhead{CPSC 585 --- Winter 2019}

\lfoot{High-level Design}
\rfoot{\team{}}

\newenvironment{hangingpar}[1]
  {\begin{list}
          {}
          {\setlength{\itemindent}{-#1}%%'
           \setlength{\leftmargin}{#1}%%'
           \setlength{\itemsep}{0pt}%%'
           \setlength{\parsep}{\parskip}%%'
           \setlength{\topsep}{\parskip}%%'
           }
    \setlength{\parindent}{-#1}%%
    \item[]
  }
  {\end{list}}


\newcommand{\sep}{\;}
\newtheorem{theorem}{Theorem}
\theoremstyle{definition}
\newtheorem{definition}[theorem]{Definition}

\newcommand\floor[1]{\lfloor#1\rfloor}
\newcommand\ceil[1]{\lceil#1\rceil}
\begin{document}
\begin{titlepage}
  \begin{center}
    \vspace*{1cm}
    \Large{\textbf{University of Calgary}}\\
    \Large{\textbf{CPSC 585 --- Winter 2019 --- Games Programming}}\\
    \vfill
    \line(1,0){400}\\[1mm]
    \huge{\textbf{\name{}}}\\
    \large{\textbf{High-Concept Design Document}}\\
    \line(1,0){400}\\
    \vfill
    \Large{\textbf{\team{}}}\\
    \Large{Austin Easton, Evan Quan, James Cote, Jianan Ding}\\
    \large{January 21, 2019}
    % \today \\
  \end{center}
\end{titlepage}
% \thispagestyle{fancy}
\setcounter{page}{0}
\tableofcontents
\pagenumbering{gobble}
\break{}
\pagenumbering{arabic}
% \onehalfspacing

\section{Game Design}

\name{} is a combat-based driving game aimed to test your skill against a team
of AI\@. The player finds themselves in an arena pitted against \botcount{}
bots to fight until the player is defeated. Utilizing abilities, power-ups, and
the navigating the map, the player must survive as long as possible to take out
the never-ending onslaught of bots.

\subsection{Game Analysis}

\name{} at its core is inspired by Mario Kart's battle mode. Elements of the
game are also inspired by Tron (1982), and Pac-Man.

\subsection{Game Concept}
\subsection{Game Goals}

\begin{itemize}
  \item The player's central goal is to gain the highest score possible before
    they lose all their lives and the game ends.
\end{itemize}

\subsection{Game Genre}
\subsection{Brand Analysis}
\subsection{Target Market}
\subsection{Competitive Analysis}

Being a single-player game, players would compete to achieve the highest score possible, either against other players, or their own past high scores.

\subsection{Gameplay Direction}
\section{Proposed Features}
\section{Game Design Elements and Terminology}
\subsection{Story}
\subsection{Terminology}
\subsection{Game Information}
\subsubsection{Arenas}
\subsubsection{Abilities}

\textbf{Movement}

\begin{itemize}
  \item \textbf{Standard movement} --- A hovercraft does not rely on wheels to
    move and so can traverse in any lateral direction without needing to turn,
    meaning that strafing is possible.
  \item \textbf{Acceleration/braking} --- A hovercraft can accelerate an brake
    in any direction is it currently moving. It will often drift if a turn is
    made, even at relatively slow speeds, which can be both advantageous and
    disadvantageous.
  \item \textbf{Dashing} --- A hovercraft can dash in any direction,
    temporarily gaining invulnerability. From a mobility standpoint, dashing
    can be used to catch up to other hovercrafts, reach power-ups faster, or
    lose others when being chased. From a defensive standpoint, it can be used
    to dodge attacks.
\end{itemize}

\textbf{Attacks}

Every hovercraft has 3 attack abilities that are available from the start of the game.

\begin{itemize}
  \item \textbf{Rocket} --- A rocket launches forward straight out from the
    direction the hovercraft is facing until it hits a surface. Upon impact, it
    explodes, damaging everything in a radius around it. Being the only ranged
    attack, it is great for attacking distant enemies if aimed well, or when
    chasing other vehicles. The splash damage can be utilized with parts of the
    arena environment to hit enemies near walls easier, or to hit multiple
    enemies that are grouped together.
  \item \textbf{Spikes} --- Spikes temporarily extend in all directions from
    the hovercraft, damaging other vehicles that come in contact with it. Can be
    used both aggressively and defensively when other vehicles are nearby. It
    can also be used in combination with dashing to crash into enemies.
  \item \textbf{Flame trail} --- A trail of fire is created that follows the
    players path. Any hovercraft that contacts it is damaged. Great to use when
    being chased.
\end{itemize}


\subsubsection{Power Ups}
\subsubsection{Difficulty}
\subsubsection{Menu}

\section{Concept Art}

\end{document}
